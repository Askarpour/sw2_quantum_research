%% LyX 2.3.3 created this file.  For more info, see http://www.lyx.org/.
%% Do not edit unless you really know what you are doing.
\documentclass[english]{article}
\usepackage[T1]{fontenc}
\usepackage[latin9]{inputenc}
\usepackage{amsmath}
\usepackage{amssymb}

\makeatletter

%%%%%%%%%%%%%%%%%%%%%%%%%%%%%% LyX specific LaTeX commands.
%% Because html converters don't know tabularnewline
\providecommand{\tabularnewline}{\\}

%%%%%%%%%%%%%%%%%%%%%%%%%%%%%% User specified LaTeX commands.
\usepackage{tikz}
\usepackage{braket}
\usepackage{Qcircuit}

\makeatother

\usepackage{babel}
\begin{document}
\title{Quantum Max Flow Analysis}
\author{Moreno Giussani}
\maketitle
\begin{abstract}
This document will describe the analysis I have performed in the last
months about the quantum implementation of the Max Flow algorithm.
The max flow problem involves finding a feasible flow through a single-source,
single-sink flow network that is maximum. I have been working for
finding a practical solution to this problem using a quantum algorithm
which could be at least as efficient as a classical one, without succeding
in it.
\end{abstract}

\part{Introduction}

Before considering the algorithm, I have tried to understand most
of the concepts which lies behind a quantum algorithm.

For what regards quantum computing, the standard model of computation
is the quantum circuit. A quantum circuit is a scheme composed of
some elementary blocks, which are qubits and quantum logic gates.
Rows of this scheme represents qubits, while in columns are inserted
quantum logic gates.

\section{Qubit}

Qubits are the quantum equivalent of bits. They are usually represented
using the bra-ket notation. A single qubit $\Ket{Q_{0}}$is usually
described by a 2-dimensional column vector (Ket nontation) which is
a specific linear combination of its orthonormal bases $\Ket{0}=\left[\begin{array}{c}
1\\
0
\end{array}\right]$ and $\Ket{1}=\left[\begin{array}{c}
0\\
1
\end{array}\right]$. When the qubit is measured (an equivalent operation of reading a
bit value in classical computing), its value collapses to either $\Ket{0}$
or $\Ket{1}$ (their orthonormal bases).

Suppose $\Ket{Q_{0}}$is defined as 
\[
\Ket{Q_{0}}=\alpha\Ket{0}+\beta\Ket{1}\;\alpha\in\mathbb{C},\beta\in\mathbb{C}
\]
Then $\alpha$ and $\beta$ must respect the rule $|\alpha|^{2}+|\beta|^{2}=1$
, because $|\alpha|^{2}$ represents the probability that a measurement
outputs $\Ket{0}$ and $|\beta|^{2}$ represents the probability that
a measurement outputs $\Ket{1}$. During the computation the qubit
can assume an ``overlapped'' state (both state 0 and state 1), but
when measured, its expressivity power reduces to a classical bit.
When both $\alpha$ and $\beta$ are different from 0 ,$Q_{0}$ is
said to be in superposition.

Qubits have also another interesting property: they cannot be copied.
There is no way to create an identical copy of an arbitrary unknown
quantum state (\textit{no cloning theorem}).

Now, things get a bit tricky when considering a N-qubit quantum computer.
If there are two or more qubits, their representation is made as the
tensor product of all of the qbits, so they often cannot be considered
as separated qubits. Suppose to have a 3-qubit quantum computer which
uses qubits $Q_{a},Q_{b},Q_{c}$. The representation of the state
of the quantum system becomes:

\begin{equation}
\Ket{Q_{x}}=x_{1}\left[\begin{array}{c}
1\\
0
\end{array}\right]+x_{2}\left[\begin{array}{c}
0\\
1
\end{array}\right]=\left[\begin{array}{c}
x_{1}\\
x_{2}
\end{array}\right]\;x\in\left\{ a,b,c\right\} 
\end{equation}

\begin{equation}
\Ket{Q_{ab}}=\Ket{Q_{a}}\otimes\Ket{Q_{b}}=\begin{bmatrix}a_{1}\begin{bmatrix}b_{1}\\
b_{2}
\end{bmatrix}\\
a_{2}\begin{bmatrix}b_{1}\\
b_{2}
\end{bmatrix}
\end{bmatrix}=\begin{bmatrix}a_{1}b_{1}\\
a_{1}b_{2}\\
a_{2}b_{1}\\
a_{2}b_{2}
\end{bmatrix}
\end{equation}

\begin{equation}
\Ket{Q_{ab}}=\begin{bmatrix}a_{1}b_{1}\\
a_{1}b_{2}\\
a_{2}b_{1}\\
a_{2}b_{2}
\end{bmatrix}=a_{1}b_{1}(\Ket{0}\otimes\Ket{0})+a_{1}b_{2}(\Ket{0}\otimes\Ket{1})+a_{2}b_{1}(\Ket{1}\otimes\Ket{0})+a_{2}b_{2}(\Ket{1}\otimes\Ket{1})
\end{equation}

\begin{equation}
\Ket{Q_{abc}}=\Ket{Q_{a}}\otimes\Ket{Q_{b}}\otimes\Ket{Q_{c}}=\Ket{Q_{ab}}\otimes\Ket{Q_{c}}=\begin{bmatrix}a_{1}b_{1}\begin{bmatrix}c_{1}\\
c_{2}
\end{bmatrix}\\
a_{1}b_{2}\begin{bmatrix}c_{1}\\
c_{2}
\end{bmatrix}\\
a_{2}b_{1}\begin{bmatrix}c_{1}\\
c_{2}
\end{bmatrix}\\
a_{2}b_{2}\begin{bmatrix}c_{1}\\
c_{2}
\end{bmatrix}
\end{bmatrix}=\begin{bmatrix}a_{1}b_{1}c_{1}\\
a_{1}b_{1}c_{2}\\
a_{1}b_{2}c_{1}\\
a_{1}b_{2}c_{2}\\
a_{2}b_{1}c_{1}\\
a_{2}b_{1}c_{2}\\
a_{2}b_{2}c_{1}\\
a_{2}b_{2}c_{2}
\end{bmatrix}
\end{equation}

In many cases it is impossible to consider $Q_{a},Q_{b}$and $Q_{c}$separately,
because in a quantum system some quantum logic gates may cause to
obtain a ``mixed'' state from which is not possible to find some
suitable $Q_{a},Q_{b}$and $Q_{c}$ which satisfies (4). This concept,
which is called entanglement, will be described in detail later. The
quadratic sum of all elements of a Ket must be 1, like said before
for a single qubit.

\section{Quantum logic gates}

In a quantum circuit model, quantum logic gates are transformation
matrices which describes the behaviour of the physical quantum logic
gates. Quantum logic gates are represented by means of unitary square
matrices of size $2^{n}$,where $n$ is the number of qubits to which
a gate can be applied. A matrix $U$ is said unitary if

\[
UU^{\dagger}=U^{\dagger}U=I
\]

where $U^{\dagger}$ is the Hermitian conjugate of $U$. The Hermitian
conjugate could be described as a conjugate traspose of $U$.

The description of the state obtained from the application of a generic
quantum gate $G$from quantum state$\Ket{S_{0}}$ can be calculated
as:

\[
\Ket{S_{1}}=G\Ket{S_{0}}
\]

Some of the most known unitary logic gates are:

\begin{center}
\begin{tabular}{|c|c|c|}
\hline 
Name & Symbol & Matrix\tabularnewline
\hline 
\hline 
Hadamard & $H$ & $\frac{1}{\sqrt{2}}\begin{bmatrix}1 & 1\\
1 & -1
\end{bmatrix}$\tabularnewline
\hline 
Pauli-X (Not) & $X$(or $NOT$) & $\left[\begin{array}{cc}
0 & 1\\
1 & 0
\end{array}\right]$\tabularnewline
\hline 
Pauli-Y & $Y$ & $\left[\begin{array}{cc}
0 & -i\\
i & 0
\end{array}\right]$\tabularnewline
\hline 
Pauli-Z & $Z$or $R_{\pi}$ & $\left[\begin{array}{cc}
1 & 0\\
0 & -1
\end{array}\right]$\tabularnewline
\hline 
Swap & $SWAP$ & $\left[\begin{array}{cccc}
1 & 0 & 0 & 0\\
0 & 0 & 1 & 0\\
0 & 1 & 0 & 0\\
0 & 0 & 0 & 1
\end{array}\right]$\tabularnewline
\hline 
Controlled Not & $CNOT$ & $\left[\begin{array}{cccc}
1 & 0 & 0 & 0\\
0 & 1 & 0 & 0\\
0 & 0 & 0 & 1\\
0 & 0 & 1 & 0
\end{array}\right]$\tabularnewline
\hline 
Identity & $I$ & $\left[\begin{array}{cc}
1 & 0\\
0 & 1
\end{array}\right]$\tabularnewline
\hline 
\end{tabular}
\par\end{center}

\begin{center}
When applied to a single qubit in one of its bases ($\Ket{0}$ or
$\Ket{1}$), an H gate will put the qubit in a superstate.
\par\end{center}

There are many more controlled gates which are represented using a
C prefix, like for the CNOT gate. Their structure is

\[
U=\left[\begin{array}{cccc}
u_{11} & u_{12} & ... & u_{1m}\\
u_{21} & u_{22} & ... & u_{2m}\\
... & ... & ... & ...\\
u_{m1} & u_{m2} & ... & u_{mm}
\end{array}\right]\:CU=\left[\begin{array}{ccccc}
1 & 0 & 0 & ... & 0\\
0 & 1 & 0 & ... & 0\\
0 & 0 & u_{11} & ... & u_{1m}\\
... & ... & ... & ... & ...\\
0 & 0 & u_{m1} & ... & u_{mm}
\end{array}\right]
\]


\section{Quantum circuits}

Quantum circuits can be represented via a ladder-like scheme. Each
row (horizontal lines) represents a distinct qubit, and the gates
which have to be applied to the given qubit are inserted on that line,
from left to right. Gates on the same column has to be applied at
the same time.

Here's an example with a 2-qubit circuit:\Qcircuit @C=1em @R=.7em {
\Ket{Q_0} & & \gate{H} & \gate{Z} & \gate{H} & \ctrl{1} & \gate{U}  & \qw & \meter \\
\Ket{Q_1} & & \qw      & \gate{X} & \qw      & \targ    & \ctrl{-1} & \qw & \meter
}

which is equivalent to the given circuit:
\begin{center}
\Qcircuit @C=1em @R=.7em {
\Ket{Q_0} & & \gate{H} & \gate{Z} & \gate{H} & \ctrl{1} & \gate{U}  & \gate{I} & \meter \\
\Ket{Q_1} & & \gate{I} & \gate{X} & \gate{I} & \targ    & \ctrl{-1} & \gate{I} & \meter
}
\par\end{center}

As said before, in a multi qubit computer, considering $Q_{0}$and
$Q_{1}$as independent qubits would often lead to mistakes, because
the application of a gate to a qubit would cause some side effects
on other qubits.

If not specified, like in this case, every qubit is initialized to
state $\Ket{0}$. 

Knowing that the above circuit is a 2-qubit circuit, the initial state
is described as $\Ket{Q_{0}}\otimes\Ket{Q_{1}}=\Ket{00}$ . Then,
the applied gate is $H\otimes I=\left[\begin{array}{cccc}
\frac{1}{\sqrt{2}} & 0 & 0 & 0\\
0 & \frac{1}{\sqrt{2}} & 0 & 0\\
0 & 0 & -\frac{1}{\sqrt{2}} & 0\\
0 & 0 & 0 & -\frac{1}{\sqrt{2}}
\end{array}\right]$
\end{document}
