\section{Coined Quantum Walk}

The following example shows how to implement a Coined discrete quantum walk on a cyclic graph with N = 8 nodes. This can be achieved using the coined DQW on a line
where the line is the cylic graph. First we need to encode the graph nodes in a binary notation to fit with qubits. In general a graph with $2^{n}$ nodes needs n encoding 
bit, in our case since we have $2^{3}$ nodes we can use 3 bit to encode. The Figure 1 below shows how we can bind the qubits to the nodes of the graph.

\begin{figure}[h!]
    \includegraphics[scale=0.3]{img/cyclic_graph.png}
    \caption{Cyclic graph with 8 nodes, and the respective position state in qubits version}
    \centering
\end{figure}

\subsection{Components description}

We can fit now the 3 main components mentioned in the previous section for this specific example. 

% walker
Starting from the \textbf{walker operator}: for this version we need to encode the nodes of the graph in 3 qubits, usually we need log(N) qubits, in general this is not always true, 
we will see why in the next section. Using 3 qubits the postion of the walker is given by the binary encoding of the node, for instance the node 0 is represented as $\ket{000}$.

% coin operator
For the \textbf{coin operator} operator, we will use an Hadamard coin, that
consists in apply the Hadamard operator to the system. 

% hadamard coin description

% shift operator
The \textbf{shift operator} operator in this case will move the actual position that we call $\ket{i}$ to one of the adjiacent nodes, that corresponds to 
$\ket{i+1}$ or $\ket{i-1}$, which is decided by the outcome of the coin operator.

\subsection{Circuit for the CQW}

To implement the circuit in qiskit we need to translate the operator defined in quantum circuits, first we need 1 qubit for the coin and 3 for the position. 
The Hadamard can be easily implemented using an Hadamard gate on the first qubit. Then we need a circuit to perform an increment and a decrement on the initial state, 
this is less trivial and to achieve that we need two sub circuit that uses multi controlled toffoli gates. The circuit below represents an incrementer circuit 
for 3 qubits.

\begin{quantikz}
    & \ket{0} & \ctrl{2} & \ctrl{1} & \targ{}  & \qw \\
    & \ket{0} & \ctrl{1} & \targ{}  & \qw      & \qw \\
    & \ket{0} & \targ{}  & \qw      & \qw      & \qw \\
\end{quantikz}

The decrement circuit is similar but uses negative controlled not gates, the circuit below shows the decrement circuit for 3 qubits.

\begin{quantikz}
    & \ket{0} & \octrl{2} & \octrl{1} & \targ{}  & \qw \\
    & \ket{0} & \octrl{1} & \targ{}  & \qw      & \qw \\
    & \ket{0} & \targ{}  & \qw      & \qw      & \qw \\
\end{quantikz}

The negative controlled not can be represented by negate before and after the controlled not, the equivalence circuit below
clarify this explanation, for a detalied explanation look at \cite{nielsen_chuang_2010}.

$$
\begin{quantikz}[baseline={($(W.base)!.5!(W2.base) -height("$\vcenter{}$")*(0,1pt)$)}]
    & \octrl{1} & \alias{W}  \qw & \qw \\
    & \targ{}   & \alias{W2} \qw & \qw 
\end{quantikz}
=\begin{quantikz}[baseline={($(W.base)!.5!(W2.base) -height("$\vcenter{}$")*(0,1pt)$)}]
    & \gate{X}  & \ctrl{1} & \gate{X} & \qw \\
    & \qw       & \targ{}  & \qw      & \qw 
\end{quantikz}
$$

Finally the decrement circuit defined above, using this equivalence, is showed below.   

\begin{quantikz}
    & \ket{0} & \targ{} & \ctrl{2} & \ctrl{1} & \targ{} & \targ{} & \qw \\
    & \ket{0} & \targ{} & \ctrl{1} & \targ{}  & \qw     & \targ{} & \qw \\
    & \ket{0} & \targ{} & \targ{}  & \qw      & \qw     & \targ{} & \qw \\
\end{quantikz}

A similar implementation in detail is covered in \cite{douglas2007efficient}.
The final circuit by combining all the components defined is showed below.

\begin{quantikz}
    &\lstick{$\ket{0}$\\coin} & \ctrl{3} & \ctrl{2} & \ctrl{1}& \qw  & \targ{} & \ctrl{3} & \ctrl{2} & \ctrl{1} & \targ{} & \qw \\
    & \ket{0} & \ctrl{2} & \ctrl{1} & \targ{} & \qw  & \targ{} & \ctrl{2} & \ctrl{1} & \targ{}  & \targ{} & \qw \\
    & \ket{0} & \ctrl{1} & \targ{}  & \qw     & \qw  & \targ{} & \ctrl{1} & \targ{}  & \qw      & \targ{} & \qw \\
    & \ket{0} & \targ{}  & \qw      & \qw     & \qw  & \targ{} & \targ{}  & \qw      & \qw      & \targ{} & \qw \\
\end{quantikz}

It's worth to make some comments about it, this circuit represents the U operator, in fact the Hadamard coin will 
randomly choose the direction to take and activate the increment or decrement sub circuit. Therefore this corresponds
to a single iteration of the walk, by successively apply this circuit we can perform a random walk on the cyclic graph.
The code for this circuit can be found in the Appendix A at the end of this document.  

