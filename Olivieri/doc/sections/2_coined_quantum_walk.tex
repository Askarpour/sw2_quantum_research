\section{Coined quantum walk: Qiskit implementation}
Now we build using Qiskit the circuit for the Quantum Walk described above for the cycle graph when N = 3. Before 
talking about code it's necessary to provide some representation of the circuit described above. What we need is 
to define how to represent a label of the graph, with N=8 we need log(N) = 3 bit, so for our quantum implementation
we need at least 3 quibit to represent the state plus 1 for the coin. Note that due to the gates needed to implement
the circuit the qubit needed can change due to the necessity of ancillary qubits for multi-toffoli gates, this will
be explained in the following section when it is explained the general case. To construct the shift operator we need 
an increment circuit and a decrement circuit. Figure 2 shows how an increment circuit is obtained, basically we need 
multicontrolled toffoli gate, the circuit below perform an increment so for instance if our actual position is 
$\ket{000}$ applyng this circuit we will obtain $\ket{001}$ this implementation is based on 

\begin{quantikz}
    & \ket{0} & \ctrl{2} & \ctrl{1} & \targ{}  & \qw \\
    & \ket{0} & \ctrl{1} & \targ{}  & \qw      & \qw \\
    & \ket{0} & \targ{}  & \qw      & \qw      & \qw \\
\end{quantikz}

the decrement operator use instead the negative controlled not, i.e. we want to 
perform the decrement operation when the outcome of the coin is zero.

\begin{quantikz}
    & \ket{0} & \targ{} & \ctrl{2} & \ctrl{1} & \targ{} & \targ{} & \qw \\
    & \ket{0} & \targ{} & \ctrl{1} & \targ{}  & \qw     & \targ{} & \qw \\
    & \ket{0} & \targ{} & \targ{}  & \qw      & \qw     & \targ{} & \qw \\
\end{quantikz}

This implementation in detail is covered in \cite{douglas2007efficient} and i based the implementation in Qiskit using \cite{garcía2007high}. Finally
the coin that we use for this example will be an Hadamard coin, thus we will apply an Hadamard gate to the first
quibit that is the one used for controll. The final circuit is shown below

\begin{quantikz}
    &\lstick{$\ket{0}$\\coin} & \ctrl{3} & \ctrl{2} & \ctrl{1}& \qw  & \targ{} & \ctrl{3} & \ctrl{2} & \ctrl{1} & \targ{} & \qw \\
    & \ket{0} & \ctrl{2} & \ctrl{1} & \targ{} & \qw  & \targ{} & \ctrl{2} & \ctrl{1} & \targ{}  & \targ{} & \qw \\
    & \ket{0} & \ctrl{1} & \targ{}  & \qw     & \qw  & \targ{} & \ctrl{1} & \targ{}  & \qw      & \targ{} & \qw \\
    & \ket{0} & \targ{}  & \qw      & \qw     & \qw  & \targ{} & \targ{}  & \qw      & \qw      & \targ{} & \qw \\
\end{quantikz}

text.

