\section{Coined Quantum walk}
This code below is referred to coined quantum walk, using Hadamard coin on a
cycle graph with number of nodes N=8.

\begin{lstlisting}[breaklines,language=Python,basicstyle=\tiny]
from qiskit import *

#increment operator for a 3-bit state register
def increment_op(circuit):
    qr = circuit.qubits
    circuit.mct([qr[0],qr[1],qr[2]],qr[3],None,mode='noancilla')
    circuit.ccx(qr[0],qr[1],qr[2])
    circuit.cx(qr[0],qr[1])
    circuit.barrier()
    return circuit

#decrement operator for a 3-bit state register
def decrement_op(circuit):
    qr = circuit.qubits
    circuit.x(qr)   
    circuit.barrier()
    circuit.mct([qr[0],qr[1],qr[2]],qr[3],None,mode='noancilla')
    circuit.ccx(qr[0],qr[1],qr[2])
    circuit.cx(qr[0],qr[1])
    circuit.barrier()
    circuit.x(qr)
    return circuit

#construct the circuit for one step of the random walk
def random_walk_step(circuit):
    #increment operator circuit
    qr_incr = QuantumRegister(4)
    increment_circ = QuantumCircuit(qr_incr, name='increment')
    increment_op(increment_circ)
    increment_inst = increment_circ.to_instruction()

    #decrement operator circuit
    qr_decr = QuantumRegister(4)
    decrement_circ = QuantumCircuit(qr_decr, name='decrement')
    decrement_op(decrement_circ)
    decrement_inst = decrement_circ.to_instruction()

    circuit.h(qr[0])
    circuit.append(increment_inst, qr[0:4])
    circuit.append(decrement_inst, qr[0:4])
    
    return circuit   

def random_walk(steps, circuit):
    for i in range(0, steps):
        random_walk_step(circuit)
    return circuit  
\end{lstlisting}

\clearpage

