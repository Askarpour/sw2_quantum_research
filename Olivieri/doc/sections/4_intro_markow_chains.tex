\section{Szegedy quantum walk introduction}

The following method proposed by Szegedy helps to reduce some limitations discussed in the previous section. In order to 
talk about Szegedy algorithm we need to define introduct some concepts.

\subsection{Markow chains}

A Markow chain is a stochastic process that consists in a sequence of random variables $X_{n}$ with $n \in \mathbb{Z}^{+}$
such that $P(X_{n} | X_{n-1}, X_{n-2},...,X_{n-N}) = P(X_{n} | X_{n-1})$. If is time-independent can be represented
by a matrix P called transition matrix. Such that the sum of each row of P is equal to 1.

\subsection{application to graphs}

We can use then a Markow Chain to perform a random walk on a graph. First, considering a graph G(V,E) we construct the adjacency matrix
as follows:

\begin{equation}
    A_{i,j} = 
    \begin{cases} 
        1 if (v_{i}, v_{j}) \in E \\
        0 otherwise
    \end{cases}
\end{equation}
