\section{Introduction}
\subsection{Random Walks}
A random walk, also known as a stochastic or random process, is a mathematical object which describe a path constitued by random step over a mathematical space. A common 
example is a random walk in the integer set Z, starting from a certain point, for instance x=0, the path is defined by randomly choose to move left or right by increase 
or decrease the actual value of x by 1. Tossing a coin will help in choosing randomly, with equal probability, the next step to take. 
The previous example could be generalized by increasing the dimension of the mathematical space considered, in a cartesian plane the starting point will have two  
coordinates and the possible moves becomes 4.
The random walk can be divided in two major classes, discrete and continous random walks, here we focus on the discrete-time, in particular we bind this to graphs
to perform the search of a marked vertex.

\subsection{Quantum Walks}

First is worth to give an idea of the random walk above in the quantum world. We can consider a cycle graph like the one in Figure 1. The nodes of the graph can be 
represented by binary labels, so that we need log(N) quibits to represents a cycle graph with N nodes. Now in order to perform a random walk on this graph we need a
way to define a shift operator, i.e. an operator that allows to move, from our current position, in one of the adjiacent nodes and a coin operator that choose randomly
the direction to take, left or right in this case. 

In the example proposed, considering the current position as $\ket{i}$ shift operator S can move the walker in the left or right position to obtain $\ket{i+1}$ or $\ket{i-1}$ 
the coin operator C will decide with probability 1/2 wheter to increment or decrement. Defined this two operators 
the quantum walk operator can be seen as U = SC, applying once this operator U correspond to a random walk step.
The states space H in which the walk operator moves is the one defined the two Hilbert spaces, one for the shift operator that we can call 
$\mathcal{H}_{p} = {|i> where i = 0...N-1 and i belonging to Z}$ and the one for coin operator called $\mathcal{H}_{c}$ and that in our case can be spanned for instance by 
{$\ket{left}$, $\ket{right}$} finally H is given by $\mathcal{H}=\mathcal{H}_{p}\otimes\mathcal{H}_{c}$. This informal explanation is trheated in details by \cite{Kempe_2003}. 

In the example we will use an Hadamard coin, that can simulate a coin toss to decide wheter increment or decrement the actual state. The behavior of the Hadamard coin 
will be described after the implementation of the example below presented.


