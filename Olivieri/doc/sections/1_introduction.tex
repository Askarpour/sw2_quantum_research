\section{Introduction}
\subsection{Random Walks}

% random walk definition
A random walk, also known as a stochastic or random process, is a mathematical object which describe a path constitued by random step over a mathematical space. A common 
example is a random walk in a line, consider the integer set Z, starting from a certain point, for instance x = 0, the path is defined by randomly choose to move left or 
right, the movement is achieved increasing or decreasing the value of x by 1. Tossing a coin will help in choosing randomly, with equal probability, the next step to take. 
This example could be generalized by increasing the dimension of the mathematical space considered, in a cartesian plane the starting point will have two  
coordinates and the possible moves becomes 4, and there are many other possible generalizations.

%random walk discrete and continous
The random walk can be divided in two major classes, discrete-time and continous-time random walk. Intuitively as the names suggest the difference between this two class 
is in the time function that could be integer or real, for a more formal definition consider the random walk as a system composed by a family of random variables ${X_{t}}$
the variables $X_{t}$ will measure the system at time t, now if we consider $t\in \mathbb{N}$  is a discrete-time stochastic process, otherwise if 
$t \in \mathbb{R}^{+} \bigcup {0}$ is a continous-time stochastic process. 

% final
Here we focus on the discrete-time, in particular we bind this to graphs considering the random walk on a cyclic graph of order N.

\subsection{Quantum Walks}

The Quantum version of the random walks is called Quantum Walks, also here there is a disctinction between the two model of discrete quantum walks and 
continous quantum walks, the focus of this research remain in the discrete-time domain also for the quantum side. 

The discrete quantum walk described here is called coined Discrete Quantum Walk on a Line (Coined DQWL), there are also versions without coin. It's worth introducing
the Coined QDWL in a formal and generic way, then apply to a more concrete example. A formal description starts considering the three main components of a Coined QDWL: 
A walker operator, a coin, an evolution operator usually called shift operator. 

% the walker
The Walker represents the position of our system and is defined in a Hilbert space inifite but countable $\mathcal{H}_{p}$, a vector in that space represents 
the postion of the walker, $\ket{position} \in \mathcal{H}_{p}$ 

% the coin
The coin operator is defined in a two-dimension Hilbert space, if we consider as basis state $\ket{0}$ and $\ket{1}$ then the coin space will becomes
$\mathcal{H}_{c} = { \ket{0}, \ket{1} }$ and with $\ket{coin} \in \mathcal{H}_{c}$.

The complete system finally will be in a Hilbert space composed by the Kronecker product of the two spaces above defined 
$\mathcal{H}=\mathcal{H}_{p}\otimes\mathcal{H}_{c}$. A state of the coined DQWL can be defined by the vector:

\begin{equation}
    \ket{\phi_{initial}} = \ket{position}_{initial} \otimes \ket{coin}_{initial}
\end{equation}

% the evolution operator
The evolution operator, also called shift operator will actually perform a step starting from the initial position, apply this operator to the
system is equal to toss a coin and depending on the outcome move the walker to the left or to the right, we can do this by increment or decrement
the actual position by 1. A possible form of the shift operator could be described by this formula:

\begin{equation}
    S = \ket{0}_{c} \bra{0} \otimes \sum{i} \ket{i + 1}_{p} \bra{i} + \ket{1}_{c} \bra{1} \otimes \sum{i} \ket{i - 1}_{p} \bra{i} 
\end{equation}

Finally we can see a walker as operator itself, called U and given by:

\begin{equation}
    U = S \times (C \otimes I_{p})
\end{equation}

Applying this operator to a given system is equal to perform a step of a random walk. A more formal explanation can be found in \cite{6812670} and \cite{Kempe_2003}.
The example on the following section apply this general concept to a cyclic graph.





