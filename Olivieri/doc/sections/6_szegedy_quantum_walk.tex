\section{Szegedy quantum walk}

The following method proposed by Szegedy helps to reduce some limitations discussed in the previous section. In order to 
talk about Szegedy algorithm we need to introduct some concepts.

\subsection{Markow chains}

A Markow chain is a stochastic process that consists in a sequence of random variables $X_{n}$ with $n \in \mathbb{Z}^{+}$
such that $P(X_{n} | X_{n-1}, X_{n-2},...,X_{n-N}) = P(X_{n} | X_{n-1})$. If is time-independent can be represented
by a matrix P called transition matrix. Such that the sum of each row of P is equal to 1.

\subsection{application to graphs}

We can use then a Markow Chain to perform a random walk on a graph. First, considering a graph G(V,E) we construct the adjacency matrix
as follows:

\begin{equation}
    A_{i,j} = 
    \begin{cases} 
        1 if (v_{i}, v_{j}) \in E \\
        0 otherwise
    \end{cases}
\end{equation}

Then we can define the transition matrix P as:

\begin{equation}
    P_{i,j} = frac{A_{i,j}}{indeg(j)}
\end{equation}

where indeg(j) represents the number of in-going arc of vertex j. Basically we can express the graph with this transition matrix, this lead to a
generalization of the quantum walks presented. The previous method uses the label encoding to represent the graph and was limited to undirected graphs
without weights. The Szegedy algorithm instead uses the matrix representation explained above, therefore there are no problem if the graph is directed
or even if has weights. 

\subsection{The szegedy QW operator}
The szegedy quantum operator