\section{Introduction}
\subsection{Random Walks}
A random walk, also known as a stochastic or random process, is a mathematical object which describe a path constitued by random step over a mathematical space. A common 
example is a random walk in the integer set Z, starting from a certain point, for instance x=0, the path is defined by randomly choose to move left or right by increase 
or decrease the actual value of x by 1. Tossing a coin will help in choosing randomly, with equal probability, the next step to take. 
The previous example could be generalized by increasing the dimension of the mathematical space considered, in a cartesian plane the starting point will have two  
coordinates and the possible moves becomes 4.
The random walk can be divided in two major classes, discrete and continous random walk: in the first class 


\subsection{Random Walks in Computer Science}
In computer science an interesting application of the object described above is in graphs. A random walk could be applied for search a specific node 
by mark and find it in random steps. %to finish

\subsection{Random walks in Quantum Computing: Quantum Walks}
Since quantum computer deals with randomness by nature we can imagine a relevant speedup on performance, for sure thanks to quantum mechanics the randomness
that help us choosing which way to take is "more random" with respect to classical computer. If we compare the speedup in time we need to make distinctions
by classes of graphs.  %to finish


