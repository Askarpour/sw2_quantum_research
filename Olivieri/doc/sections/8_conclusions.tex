\section{Conclusions}

In this research we presents the analogue of random walks, the quantum walks in discrete time with a focus on graph applications. We start from 
an example of Coined quantum walk on a cycle walk to shows some characteristics and difference w.r.t. classical of this method. This was just
an example, in fact this method could be generalized for other types of graphs, or for other version without coins. Then we presents 
an analysis of this method applied to the problem of search a marked vertex for some family of graphs with quadratic speedup w.r.t. classical
and using an efficient circuit. The however method presents some limitations, it works only for undirected and not weighted graph. The Szegedy
quantum walk helps to get rid of limitations using a markow chains to repsents the graph without express it with encoded label. This method
can be efficiently implemented and could be used for interesting applications showed in the prevous section.    