\section{Conclusions}

This research presents the analogue of random walks, the quantum walks in discrete time with a focus on graph applications. We started from 
an example of Coined quantum walk on a cycle walk to show some characteristics and difference w.r.t. classical of this method. This was just
an example, in fact this method could be generalized for other types of graphs, or for other version without coins. Then we performed
an analysis of this method applied to the problem of search a marked vertex for some family of graphs thanks to \cite{douglas2014complexity}
we know that quadratic speedup w.r.t. classical, using an efficient circuit. However the method presents some limitations, it works only for undirected and not weighted graph. The Szegedy
quantum walk helps to get rid of limitations by quantiziong the markow chain related to the graph. This method
can be efficiently implemented for the cycle graph. Some personal though about this model of computation,    